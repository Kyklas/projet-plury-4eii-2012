% !TEX TS-program = pdflatex
% !TEX encoding = UTF-8 Unicode

%%% DOCUMENT DEFINITION
\documentclass[11pt, french]{article} % use larger type; default would be 10pt
\usepackage[utf8]{inputenc} % set input encoding (not needed with XeLaTeX)

%%% PAGE DIMENSIONS
\usepackage{geometry} % to change the page dimensions
\geometry{a4paper} % or letterpaper (US) or a5paper or....
\geometry{margin=1in} % for example, change the margins to 2 inches all round

%%% PACKAGES
\usepackage{graphicx} % support the \includegraphics command and options
\usepackage{booktabs} % for much better looking tables
\usepackage{array} % for better arrays (eg matrices) in maths
\usepackage{paralist} % very flexible & customisable lists (eg. enumerate/itemize, etc.)
\usepackage{verbatim} % adds environment for commenting out blocks of text & for better verbatim
\usepackage{subfig} % make it possible to include more than one captioned figure/table in a single float
\usepackage[frenchb]{babel}

% Package pour le dessin schéma bloc
\graphicspath{{../Automatique/}}
\usepackage{schemabloc}
\usepackage{amsmath}

%%% HEADERS & FOOTERS
%\usepackage{fancyhdr} % This should be set AFTER setting up the page geometry
%\pagestyle{fancy} % options: empty , plain , fancy

% Rapport projet pluridisciplinaire
% : Xavier Galzin, Stanislas Bertrand, Romain Desille, Frédéric Meslin

\title{Projet pluridisciplinaire : sustentation magnétique}
\author{ Xavier GALZIN, Stan BERTRAND, Romain DESILLE, Fred MESLIN}
\date{30/01/2012}


\begin{document}
\maketitle

\begin{center}
	\vspace{0.6in}
	\includegraphics[width=10cm]{Automatique/system_physique.png} 
	\\
	\emph{Le système physique étudié}
\end{center}

\pagebreak

\section{Introduction}
\paragraph{}
	Dans le cadre du projet pluridisciplinaire de cette année, il nous est demandé de faire léviter un mobile dans l'air à l'aide d'un champ magnétique produit par une bobine. On appelle cet effet la sustentation magnétique. Ce projet est une version édulcorée d'un prototype de lampe décorative imaginée par une étudiante de l'école LISAA (Institut supérieur des arts appliqués) et réalisée en partenariat avec des chercheurs et ingénieurs de l'INSA de Rennes.

L'étude demandée présente de nombreuses facettes : il faudra étudier et modéliser le système physique pour ensuite l'asservir à l'aide d'un dispositif d'automatique électronique. Ce dispositif sera réalisé de manière analogique puis de manière numérique. Dans ce rapport, on s'intéressera uniquement à la description du système, sa modélisation et ses constantes prépondérantes ainsi qu'aux correcteurs que nous avons choisi d'implémenter pour maintenir le mobile en lévitation.

\section{Description du système physique}
\paragraph{}

Le système physique est existant et imposé. Il consiste en une potence munie d'une bobine qui servira à exercer le champs magnétique retenant le mobile. La bobine est également munie de capteurs à effet Hall qui permettent, par une mesure différentielle, de mesurer la distance du mobile par rapport à la bobine. On a ainsi la boucle de retour de notre système. 

On a donc finalement une consigne en position que l'on va convertir en un courant qui servira à générer le champs magnétique qui attirera ou repoussera le mobile et le retour en position par le biais des capteurs à effet Hall. 

Le problème principal du système est qu'il n'est pas stable en boucle ouverte. On va donc procéder à une étude basée sur des mesures fournies afin de mettre un place un correcteur qui rendra la système stable. Pour cela, nous poserons les équations mathématiques du problème pour ensuite décrire le modèle de correcteur retenu ainsi que les valeurs des diverses constantes associées à ce modèle.

\section{Modélisation mathématique}

\paragraph{Mise en équation du système}

Forces appliquées sur le système :
\newline

$ F_{bobine} = K \times \dfrac{i^2}{x^2}  $

$ F_{poids} = -m \times g $

On applique le principe de la dynamique :
\newline

$ \sum F_{exterieures}  = m \times {\dfrac{{d^2}x}{dt^2}} $

Linéarisation de la force exercée par la bobine :
\newline

$ {\dfrac{dF(t)}{dt}}_{x = X_0, i = I_0} = \dfrac{\partial F(t)}{\partial x} \times dx + \dfrac{\partial F(t)}{\partial i} \times di $

$ \dfrac{\partial F(t)}{\partial x} = K_x = \dfrac{-2K \times I_0^2}{X_0^3} $

$ \dfrac{\partial F(t)}{\partial i} = K_i = \dfrac{2K \times I_0}{X_0^2} $

Equation différentielle résultante :
\newline

$ m \times {\dfrac{{d^2}x}{dt^2}} = F_{bobine} - m \times g = K \times {\dfrac{I_0^2}{X_0^2}} + K_x \times Dx + K_i \times Di - m \times g $

$ avec : Dx = x - X_0, Di = i - I_0 $

On effectue une transformée de Laplace du système :
\newline

$ m \times x{s^2} = F_0 + K_x \times (x - X_0) + K_i \times (i - I_0) - m \times g $

$ x \times (m{s^2} - K_x) = K_i \times i +(F_0 - K_x \times X_0 - K_i \times I_0 - m \times g) $

$ H(s) = \dfrac{x(s)}{i(s)} = \dfrac{K_i}{m{s^2}- K_x}  $

Etude du point d'équilibre :
\newline

$ \sum F_{exterieures}  = K \times {\dfrac{I_0^2}{X_0^2}} - m \times g = 0 $

$ K = mg \times {\dfrac{X_0^2}{I_0^2}} $

\paragraph{Constantes obtenues par le calcul}

$ K_{bobine} = 1,459 \times 10^{-4} $
$ K_x = 141,7 $
$ K_i = 1,660 $

\section{Correcteur analogique}
\subsection{Choix du correcteur}
Dans un premier temps, nous avons choisi un correcteur proportionnel-dérivé. Un correcteur dérivé n'étant pas réalisable physiquement, nous allons donc le remplacer par un correcteur à avance de phase.

Pourquoi l'avance de phase ?

\vspace{0,2in}
\begin{tikzpicture}
\sbEntree{Vxc}
\sbComp{Comp}{Vxc}
	\sbRelier[$Vx_C$]{Vxc}{Comp}
\sbBloc{C}{$K_{p}$}{Comp}
	\sbRelier[$\epsilon_x$]{Comp}{C}
\sbBlocL{E}{$\dfrac{K_{Elec}}{\tau_{Elec} \cdot p + 1}$}{C} 
\sbBlocL{M}{$\dfrac{-\frac{K_i}{K_x}}{-{\tau_{Meca}}^2 \cdot p^2 - 1}$}{E}
\sbBlocL{H}{$K_{Hall}$}{M} 

\sbSortie{Vxr}{H} 
	\sbRelier{H}{Vxr}
	\sbNomLien[0.8]{Vxr}{$Vx_R$} 
\sbDecaleNoeudy{M}{A}
\sbBlocr[0]{A}{$\dfrac{1+ \tau_d \cdot p}{1 + a_d \cdot \tau_d \cdot p}$}{A}
\sbRelieryx{H-Vxr}{A}
\sbRelierxy{A}{Comp}
\end{tikzpicture}




\subsection{Choix des constantes}

Pour choisir les constantes du correcteur, nous avons procédé en deux phases. Premièrement, nous avons réglé le correcteur à avance de phase. Pour cela, nous nous sommes servis du lieu d'Evans. On peut remarquer que, quelque soit le gain du correcteur, au moins un des poles du systeme restera à droite de l'axe des imaginaires.

\newline
\begin{center}
\includegraphics [scale=0.50]{RL_Sys_Seul.pdf}
\\
\emph{Lieu d'Evans du Systeme non corrigé}
\end{center}

L'idée est donc de décaler le point d'intersection des asymptotes qui est normalement dans la partie droite du plan complexe en placant le zéros du correcteur à avance de phase dans la partie gauche tout en maintenant ce zéro avant le pole issu de la constante de temps mécanique qui se trouve dans la partie gauche. Une fois que la position des asymptotes a été correcte, nous avons réglé le gain du correcteur proportionnel afin que tous les poles du systeme se retrouvent à droite de l'axe des imaginaires. Nous avons trouvé un gain K de 5. 

\newline
\begin{center}
\includegraphics[scale=0.50]{RL_Sys_AvPh_K5.pdf}
\\
\emph{Lieu d'Evans du systeme corrigé}
\end{center}

Nous avons ensuite tracé le diagramme de Black-Nichols du système corrigé afin de vérifier les marges de gain et de phase. Cela nous permet d'obtenir une marge de phase d'environ 25° et une marge de gain de plus de 10 dB.

\newline
\begin{center}
\includegraphics[scale=0.50]{MatBlack.pdf}
\\
\emph{Lieu d'Evans du systeme corrigé}
\end{center}


\section{Correcteur numérique}

Dans la seconde partie du projet, nous commencerons par simplement convertir notre correcteur analogique par le biais d'une approximation comme celle de Tustin ou celle des rectangles avant. Une fois que cette solution fonctionnera, nous tenterons d'implémenter un modèle non linéaire en nous servant des équations du système. 

\includegraphics[scale=0.50]{RLN_Sys_Seul.pdf}
\includegraphics[scale=0.50]{RLN_Sys_AvPh_K5.pdf}


\section{Conclusion}

Grâce à cette étude, nous avons pu trouver des valeurs de paramètres de correcteurs qui nous permettent de stabiliser le système. Cependant, nous pouvons nous interroger sur la fiabilité de ces paramètres étant donné qu'ils sont basés sur une approximation linéaire du système. De même, le modèle du champs magnétique pourrait se révéler trop imprécis et ainsi fausser le calcul des constantes. Il faudra donc garder cela à l'esprit pour éventuellement réévaluer nos paramètres par la suite. 



\end{document}
