% !TEX TS-program = pdflatex
% !TEX encoding = UTF-8 Unicode

%%% DOCUMENT DEFINITION
\documentclass[11pt, french]{article} % use larger type; default would be 10pt
\usepackage[utf8]{inputenc} % set input encoding (not needed with XeLaTeX)

%%% PAGE DIMENSIONS
\usepackage{geometry} % to change the page dimensions
\geometry{a4paper} % or letterpaper (US) or a5paper or....
\geometry{margin=1in} % for example, change the margins to 2 inches all round

%%% PACKAGES
\usepackage{graphicx} % support the \includegraphics command and options
\usepackage{booktabs} % for much better looking tables
\usepackage{array} % for better arrays (eg matrices) in maths
\usepackage{paralist} % very flexible & customisable lists (eg. enumerate/itemize, etc.)
\usepackage{verbatim} % adds environment for commenting out blocks of text & for better verbatim
\usepackage{subfig} % make it possible to include more than one captioned figure/table in a single float
\usepackage[frenchb]{babel}
\usepackage{pdflscape}
\usepackage{rotating}

% Package pour schémas MCSE
\usepackage{tikz}
\usetikzlibrary{calc,backgrounds}
\usetikzlibrary{arrows,positioning} 
\usetikzlibrary{decorations,decorations.pathmorphing} 
\usetikzlibrary{shapes,snakes,fit}


% Schémas individuels MCSE
\graphicspath{{../MCSE/}}

%%% HEADERS & FOOTERS
%\usepackage{fancyhdr} % This should be set AFTER setting up the page geometry
%\pagestyle{fancy} % options: empty , plain , fancy

% Rapport projet pluridisciplinaire
% : Xavier Galzin, Stanislas Bertrand, Romain Desille, Frédéric Meslin

\title{Projet pluridisciplinaire : sustentation magnétique}
\author{ Xavier GALZIN, Stan BERTRAND, Romain DESILLE, Fred MESLIN}
\date{30/01/2012}

\begin{document}
\maketitle
\pagebreak

\shorthandoff{:}
\newcommand{\Etape}[2]{
\node[Etape,label={right:#2}] (#1) {};
}
\newcommand{\EtapeLink}[3]{
\node[Etape,below of=#3,label={[align=left]right:#2}] (#1) {};
\draw[->] (#3) -- (#1);
}
\newcommand{\TransLink}[3]{
\node[Trans,align=center,below of=#3] (#1) {};
\node[align=center,anchor=west,at =(#1.east),xshift=10pt] {#2};
\draw[->] (#3) -- (#1);
}
\newcommand{\TransActionLink}[3]{
\node[Trans,align=center,below of=#3] (#1) {};
\node[Line,align=center,anchor=west,at =(#1.east),xshift=10pt] {#2};
\draw[->] (#3) -- (#1);
}



	\tikzset{every node/.style={font=\small, node distance =  0.6cm}}
	\tikzset{Etape/.style={draw,black,minimum size=0.6cm,inner sep=0pt,circle,font=\tiny}}
	\tikzset{Trans/.style={draw,black,minimum width=0.6cm,minimum height=0.0cm,inner sep=0pt,outer sep=0pt,rectangle,font=\tiny}}
	
	%\tikzset{Line/.style={path picture={\draw[black] (path picture bounding box.north west) -- (path picture bounding box.north east);}}}
	\tikzset{Line/.style={path picture={\draw[black] (path picture bounding box.west) -- (path picture bounding box.east);}}}
\part{Spécification fonctionnelle}

\section{Analyse de l'environnement}

\paragraph{} Dans cette partie, nous allons expliciter la décomposition en entités que nous avons retenue.
\vspace{0.2in}

\begin{minipage}[t]{9cm}
\hspace{0.2in}
\underline{Convertisseur Analogique-Numérique}
\begin{description}
\item[Sortie :] Sample, information, entier
\item[Sortie :] S\_Available, évènement
\item[Entrée :] S\_Query, évènement
\end{description}

\shorthandoff{:}
\newcommand{\Etape}[2]{
\node[Etape,label={right:#2}] (#1) {};
}
\newcommand{\EtapeLink}[3]{
\node[Etape,below of=#3,label={right:#2}] (#1) {};
\draw[->] (#3) -- (#1);
}
\newcommand{\TransLink}[3]{
\node[Trans,align=center,below of=#3] (#1) {};
\node[align=center,anchor=west,at =(#1.east),xshift=10pt] {#2};
\draw[->] (#3) -- (#1);
}
\newcommand{\TransActionLink}[3]{
\node[Trans,align=center,below of=#3] (#1) {};
\node[Line,align=center,anchor=west,at =(#1.east),xshift=10pt] {#2};
\draw[->] (#3) -- (#1);
}




	\tikzset{Etape/.style={draw,black,minimum size=0.7cm,inner sep=0pt,circle,font=\tiny}}
	\tikzset{Trans/.style={draw,black,minimum width=0.7cm,minimum height=0.0cm,inner sep=0pt,outer sep=0pt,rectangle,font=\tiny}}
	
	%\tikzset{Line/.style={path picture={\draw[black] (path picture bounding box.north west) -- (path picture bounding box.north east);}}}
	\tikzset{Line/.style={path picture={\draw[black] (path picture bounding box.west) -- (path picture bounding box.east);}}}

\begin{tikzpicture}\shorthandoff{:}

\Etape{I}{}
\TransLink{T1}{S\_query}{I}
\EtapeLink{S}{Sampling}{T1}
\TransActionLink{T2}{Fin Sampling\\Sample:=Data, S\_available}{S}
\draw[->] (T2) edge [bend left=160,distance=2.5cm] (I);


\end{tikzpicture}









\end{minipage}
~
\begin{minipage}[t]{8cm}
\hspace{0.2in}
\underline{Pilotage Bobine (PWM)}
\begin{description}
\item[Entrée : ] Out, donnée permanente, entier
\end{description}

\begin{tikzpicture}

\node[Etape,label=left:Génétation PWM] (I) {};

\node[Trans,node distance = 2cm,align=center,right of=I] (T) {};
\node[Line,align=center,anchor=west,at =(T.east),xshift=10pt] {Volonté de Rafraichir\\PWM:=Out};

\draw[->] (I) edge [bend left=90] (T);
\draw[->] (T) edge [bend left=90] (I);


\end{tikzpicture}
\end{minipage}

\begin{minipage}[h]{20cm}
\hspace{0.2in}
\hspace{0.2in}
\underline{Utilisateur}   
\begin{description}
\item[Sortie : ] Rx, donnée permanente, bit
\item[Entrée : ] Tx, donnée permanente, bit
\end{description}

\begin{tikzpicture}
\Etape{I}{$T_x:=Bit_{Stop}$}
\TransLink{T1}{Volonté d'envoyer}{I}
\EtapeLink{S}{$T_x:=Bit_{Start}$}{T1}
\TransActionLink{T2}{$\uparrow$ I$_{clk}$ \\ $i:=8$}{S}
\EtapeLink{D}{\emph{Bit de donnée}}{T2}
%\tikzstyle{every label}={draw,red,font=\tiny}

\node[Trans,node distance = 1.5cm,left of=D] (T) {};
\node[Line,align=center,anchor=east,at =(T.west),xshift=-10pt] (V) {$\uparrow$ I$_{clk}$ $\cdot$ $i>0$\\[1em]$T_x:=T_{Data}[i]$, $i$-\,- };
\draw[->] (D) edge [out=225,in=-90] (T);
\draw[->] (T) edge [out=90,in=135] (D);

\TransLink{T3}{$\uparrow$ I$_{clk}$ $\cdot$ $i=0$}{D}
\EtapeLink{S}{$T_x:=Bit_{Stop}$}{T3}
\TransActionLink{T4}{$\uparrow$ I$_{clk}$ \\ /Fin Transmition}{S}

\draw[-] (T4) edge [out=-90,in=-90] ($(V.west)-(10pt,0)$);
\draw[->] ($(V.west)-(10pt,0)$) edge [out=90] (I);
\end{tikzpicture}

% reception des données

\begin{tikzpicture}%[every node/.style={font=\tiny}]
\Etape{I}{Attente}
\TransActionLink{T1}{$\uparrow$ I$_{clk}$ $\cdot$ $R_x=Bit_{Start}$ \\ $i:=8$ }{I}
\EtapeLink{D}{\emph{Bit de donnée}}{T1}
%\tikzstyle{every label}={draw,red,font=\tiny}

\node[Trans,node distance = 1.5cm,left of=D] (T) {};
\node[Line,align=center,anchor=east,at =(T.west),xshift=-10pt] (V) {$\uparrow$ I$_{clk}$ $\cdot$ $i>0$\\[1em]$R_{Data}[i]:=R_x$, $i$-\,- };
\draw[->] (D) edge [out=225,in=-90] (T);
\draw[->] (T) edge [out=90,in=135] (D);

\TransLink{T3}{$\uparrow$ I$_{clk}$ $\cdot$ $i=0$}{D}
\EtapeLink{S}{Attente}{T3}
\TransActionLink{T4}{$R_x=Bit_{Stop}$\\ /Fin Réception}{S}

\draw[-] (T4) edge [out=-90,in=-90] ($(V.west)-(10pt,0)$);
\draw[->] ($(V.west)-(10pt,0)$) edge [out=90] (I);
\end{tikzpicture}

% Schema de communication

\begin{tikzpicture}%[every node/.style={font=\tiny}]

\node (T) {};

\TransLink{T1}{Volonté Lecture}{T}

\EtapeLink{E1}{$T_{Data}:=Commande$, /Volonté d'envoyer}{T1}

\TransLink{T2}{Fin Transmition}{E1}

\EtapeLink{E2}{Lecture}{T2}

\TransActionLink{T3}{Fin Réception \\ $R_{Data}$}{E2}

\EtapeLink{E3}{}{T3}

	\draw[-] (E3) edge [out=160, in=90] (T.south);


\TransLink{T1}{Volonté Ecriture}{E3}

\EtapeLink{E1}{$T_{Data}:=Commande$, /Volonté d'envoyer}{T1}

\TransLink{T2}{Fin Transmition}{E1}

\EtapeLink{E2}{$T_{Data}:=Donnee$, /Volonté d'envoyer}{T2}

\TransLink{T3}{Fin Transmition}{E2}

	\draw[->] (T3) edge [out=-90, in=210,out distance=4cm] (E3);

\end{tikzpicture}

\end{minipage}

\newpage

Les fonctions de Transmition/Reception, sont respectivement representée avec $T_x$ et $R_x$, or les deux péripheriques de communication série sont reliés via un cable croisé. Le $T_x$ de l'utilisateur est donc en realité le $R_x$ du systeme et inversement.

\begin{minipage}[t, h]{5cm}
\hspace{0.2in}
\underline{Fonction Transmition}
\end{minipage}
~
\begin{minipage}[t, h]{9cm}

\begin{tikzpicture}
\Etape{I}{$T_x:=Bit_{Stop}$}
\TransLink{T1}{Volonté d'envoyer}{I}
\EtapeLink{S}{$T_x:=Bit_{Start}$}{T1}
\TransActionLink{T2}{$\uparrow$ I$_{clk}$ \\ $i:=8$}{S}
\EtapeLink{D}{\emph{Bit de donnée}}{T2}
%\tikzstyle{every label}={draw,red,font=\tiny}

\node[Trans,node distance = 1cm,left of=D] (T) {};
\node[Line,align=center,anchor=east,at =(T.west),xshift=-10pt] (V) {$\uparrow$ I$_{clk}$ $\cdot$ $i>0$\\[1em]$T_x:=T_{Data}[i]$, $i$-\,- };
\draw[->] (D) edge [out=225,in=-90] (T);
\draw[->] (T) edge [out=90,in=135] (D);

\TransLink{T3}{$\uparrow$ I$_{clk}$ $\cdot$ $i=0$}{D}
\EtapeLink{S}{$T_x:=Bit_{Stop}$}{T3}
\TransActionLink{T4}{$\uparrow$ I$_{clk}$ \\ /Fin Transmition}{S}

\draw[-] (T4) edge [out=-90,in=-90] ($(V.west)-(10pt,0)$);
\draw[->] ($(V.west)-(10pt,0)$) edge [out=90] (I);
\end{tikzpicture}
\end{minipage}

\begin{minipage}[t, h]{5cm}
\hspace{0.2in}
\underline{Fonction Reception}
\end{minipage}
~
\begin{minipage}[t, h]{9cm}
% reception des données

\begin{tikzpicture}[every node/.style={font=\footnotesize, node distance =  0.7cm}]
\Etape{I}{Attente}
\TransActionLink{T1}{$\uparrow$ I$_{clk}$ $\cdot$ $R_x=Bit_{Start}$ \\ $i:=8$ }{I}
\EtapeLink{D}{\emph{Bit de donnée}}{T1}
%\tikzstyle{every label}={draw,red,font=\tiny}

\node[Trans,node distance = 1cm,left of=D] (T) {};
\node[Line,align=center,anchor=east,at =(T.west),xshift=-10pt] (V) {$\uparrow$ I$_{clk}$ $\cdot$ $i>0$\\[1em]$R_{Data}[i]:=R_x$, $i$-\,- };
\draw[->] (D) edge [out=225,in=-90] (T);
\draw[->] (T) edge [out=90,in=135] (D);

\TransLink{T3}{$\uparrow$ I$_{clk}$ $\cdot$ $i=0$}{D}
\EtapeLink{S}{Attente}{T3}
\TransActionLink{T4}{$R_x=Bit_{Stop}$\\ /Fin Réception}{S}

\draw[-] (T4) edge [out=-90,in=-90] ($(V.west)-(10pt,0)$);
\draw[->] ($(V.west)-(10pt,0)$) edge [out=90] (I);
\end{tikzpicture}
\end{minipage}

\section{Délimitation des E/S du système}

\begin{tikzpicture}\shorthandoff{:}

\node[inner sep=1cm,rounded corners,fill=gray!50]
(SYS)
{Système};

\node[above of=SYS,node distance=4cm,ellipse,decoration={bumps,mirror},decorate,draw,align=center] 
(CONV)
{Convertisseur \\ Analogique-Numérique};

\node[left of=SYS,node distance=6cm,ellipse,decoration={bumps,mirror},decorate,align=center,draw] 
(BOB)
{Pilotage Bobine \\ Pulse Width Modulation};

\node[right of=SYS,node distance=6cm,ellipse,decoration={bumps,mirror},inner sep=0.3cm,decorate,draw]
(USER)
{Utilisateur};

\draw [->>,very thick] (SYS) -- node[above,pos=0.5]{Out} (BOB) ;

\draw [<-, thick] (CONV) edge[dashed,bend right=35] node[left,pos=0.5]{S\_Qerry} (SYS) ;
\draw [->>,very thick] (CONV) edge[bend right=15] node[right,pos=0.5]{Sample} (SYS) ;
\draw [->, thick] (CONV) edge[dashed,bend left=35] node[right,pos=0.5]{S\_available} (SYS) ;




\draw [->>,very thick] (SYS) edge[bend left=10] node[above,pos=0.5]{Tx} (USER) ;
\draw [<<-,very thick] (SYS) edge[bend left=-10] node[below,pos=0.5]{Rx} (USER) ;


\end{tikzpicture}

\section{Spécification fonctionelle du système}

Le système sera découpé en trois bloc :
\vspace{0.1in}
\begin{description}
\item[Correction - Asservissement : ] Aquisition des données, Correction et Pilotage du PWM.
\item[Horloge : ] Gestion de différent Temps et envoie des événement appropriés.
\item[Gestion Communication : ] Gestion de la communication avec l'utilisateur.
\end{description}

\vspace{0.2in}
\underline{Correction - Asservissement}

\begin{center}
\vspace{1cm}
\begin{tikzpicture}[every node/.style={font=\footnotesize, node distance =  0.7cm}]
\shorthandoff{:}
\node[Etape,label=right:Attente] (I) {};
\TransActionLink{T1}{$Te_{clk}$\\/S\_query}{I}
\EtapeLink{S}{Attente Sampling}{T1}
\TransActionLink{T2}{S\_available\\S[0]=S[0]+Sample, $i:=i+1$}{S}
\node[Trans,node distance = 1cm, align=center,left of=S,label=left:$i<Nb_{Acq}$] (T4) {};
\EtapeLink{F}{Acquition de $Nb_{Acq}$}{T2}

\draw[->] (F) edge [bend left] (T4);
\draw[->] (T4) edge [bend left] (I);

\TransActionLink{T3}{$i=Nb_{Acq}$\\$i:=0$, S[0]=S[0]$\div Nb_{Acq}$}{F}
\EtapeLink{C}{Correction}{T3}
\TransActionLink{T5}{$true$\\$Out := f(Out_1, S[0], S[1], Consigne, Td, Te, Kp)$}{C}
\EtapeLink{C1}{Vieillissement des variables}{T5}
\TransActionLink{T6}{$true$\\$Out_1 := Out$, S[1]=S[0]}{C1}
\draw[-] (T6) edge [out=-90,in=-90] ($(T4.west)-(70pt,0)$);
\draw[->] ($(T4.west)-(70pt,0)$) edge [out=90] (I);





\end{tikzpicture}
\end{center}

\vspace{0.2in}
\underline{Horloge}

\begin{minipage}[t, h]{8cm}
\vspace{-1.5cm}
\begin{tikzpicture}\shorthandoff{:}

\node[Etape,label=right:Attente Te] (I) {};
\TransActionLink{T}{Nouvel Echantillonage\\$Te_{clk}$}{I}
\draw[->] (T) edge [bend left=160,distance=2.5cm] (I);

\end{tikzpicture}
\end{minipage}
~
\begin{minipage}[t, h]{8cm}
\vspace{-1.5cm}
\begin{tikzpicture}\shorthandoff{:}

\node[Etape,label=right:Attente Periode UART] (I) {};
\TransActionLink{T}{Nouvelle Periode UART\\$I_{clk}$}{I}
\draw[->] (T) edge [bend left=160,distance=2.5cm] (I);

\end{tikzpicture}
\end{minipage}


\underline{Gestion Communication}
\vspace{0.2in}

La partie Gestion communication avec l'utilisateur est séparer en 3 fonction. La fonction "Principale" qui lance les fonctions d'écriture ou de lecture suivant la valeur de RW.

%\vspace{0.2in}
%\hspace{0.2in}
%\underline{Modification}
%
%\begin{center}
%
\begin{tikzpicture}\shorthandoff{:}

\node[Etape,double,label={[align=left]right:Modif}] (I) {};


\TransActionLink{T1}{$Cmd.AP = 1$\\ }{I}

\node[align=left,anchor=north west,at =(T1.south east),xshift=10pt] {$K_p := SK_p$\\$T_dT_d$\\$T_e := ST_e$\\$Consigne := SConsigne$\\$Nb_{Acq} := SNb_{Acq}$};


\node[Trans,node distance = 1.1cm, align=center,left of=T1,label=left:{$Cmd.AP = 0$}] (T2) {};

\node[Etape,double,below of=I,node distance=4.5cm,label={[align=left]right:Modif End}] (E) {};

\draw[->] (I) edge [in=90,out=200] (T2);
\draw[->] (T2) edge[out=-90] (E);

\draw[->] (T1) edge (E);


\end{tikzpicture}
%\end{center}

\vspace{0.2in}
\hspace{0.2in}
\underline{Gestion Communication (Principale)}

\begin{center}
\vspace{0.2in}
\begin{tikzpicture}[every node/.style={font=\footnotesize, node distance =  0.7cm}]
\shorthandoff{:}

\node[Etape,label=right:Attente Commande] (I) {};
\TransActionLink{T}{Fin reception\\$Cmd := R_{Data}$}{I}
\EtapeLink{E1}{}{T1}

\draw (E1) -- ($(E1)+(-5,0)$) node[inner sep=0,outer sep=0] (E10) {};
\TransLink{T10}{$RW = Lecture$}{E10}
\EtapeLink{E11}{Mode Lecture}{T10}

\draw (E1) -- ($(E1)+(5,0)$) node[inner sep=0,outer sep=0] (E20) {};
\TransLink{T20}{$RW = Ecriture$}{E20}
\EtapeLink{E21}{Mode Ecriture}{T20}

\end{tikzpicture}
\end{center}

\pagebreak
\begin{landscape}

\hspace{0.2in}
\underline{Mode Lecture}

\begin{center}
\vspace{-1cm}
\begin{tikzpicture}[every node/.style={font=\footnotesize, node distance =  0.9cm}]
\shorthandoff{:}

\node[Etape,label=right:Mode Lecture] (I) {};
\TransLink{T}{}{I}

\EtapeLink{I2}{}{T}

\draw (I2) -- ($(I2)+(-11,0)$) node[inner sep=0,outer sep=0] (E10) {};
\TransLink{T10}{$Cmd.P_{[3..0]} = 0 (Aucun) + $\\$Cmd.P_{[3..0]} > 5$}{E10}
\EtapeLink{E11}{Attente\\Commande}{T10}

\draw (I2) -- ($(I2)+(-6,0)$) node[inner sep=0,outer sep=0] (E20) {};
\TransLink{T20}{$Cmd.P_{[3..0]} = 1 $\\$(K_p)$}{E20}
\EtapeLink{E21}{$T_{Data}:= K_p$ \\ /Volonté d’envoyer}{T20}
\TransLink{T21}{Fin Transmition}{E21}
\EtapeLink{E22}{Attente\\Commande}{T21}

\draw (I2) -- ($(I2)+(-1.7,0)$) node[inner sep=0,outer sep=0] (E30) {};
\TransLink{T30}{$Cmd.P_{[3..0]} = 2 $\\$(T_d$)}{E30}
\EtapeLink{E31}{$T_{Data}:= T_d$ \\ /Volonté d’envoyer}{T30}
\TransLink{T31}{Fin Transmition}{E31}
\EtapeLink{E32}{Attente\\Commande}{T31}

\draw (I2) -- ($(I2)+(+2.6,0)$) node[inner sep=0,outer sep=0] (E40) {};
\TransLink{T40}{$Cmd.P_{[3..0]} = 3 $\\$(T_e)$}{E40}
\EtapeLink{E41}{$T_{Data}:= T_e$ \\ /Volonté d’envoyer}{T40}
\TransLink{T41}{Fin Transmition}{E41}
\EtapeLink{E42}{Attente\\Commande}{T41}

\draw (I2) -- ($(I2)+(+6.9,0)$) node[inner sep=0,outer sep=0] (E50) {};
\TransLink{T50}{$Cmd.P_{[3..0]} = 4 $\\$(Consigne)$}{E50}
\EtapeLink{E51}{$T_{Data}:= Consigne$ \\ /Volonté d’envoyer}{T50}
\TransLink{T51}{Fin Transmition}{E51}
\EtapeLink{E52}{Attente\\Commande}{T51}

\draw (I2) -- ($(I2)+(+12,0)$) node[inner sep=0,outer sep=0] (E60) {};
\TransLink{T60}{$Cmd.P_{[3..0]} = 5 $\\$(Nb_{Acq})$}{E60}
\EtapeLink{E61}{$T_{Data}:= Nb_{Acq}$ \\ /Volonté d’envoyer}{T60}
\TransLink{T61}{Fin Transmition}{E61}
\EtapeLink{E62}{Attente\\Commande}{T61}

\end{tikzpicture}
\end{center}

\hspace{0.2in}
\vspace{0.2in}
\underline{Mode Ecriture}

\begin{center}
\vspace{-1cm}
\begin{tikzpicture}[every node/.style={font=\footnotesize, node distance =  0.9cm}]
\shorthandoff{:}

%\node[Etape,label=right:Mode Ecriture] (I) {};
\Etape{I}{Mode Ecriture}
\TransLink{T}{Fin Réception}{I}
\EtapeLink{I2}{}{T}

\node[Etape,double,below of=I2,node distance=2.5cm,label={[align=left]above right:Modif}] (MODIF) {};
\TransLink{T21}{1}{MODIF}
\EtapeLink{E22}{Attente\\Commande}{T21}

\draw (I2) -- ($(I2)+(-11,0)$) node[inner sep=0,outer sep=0] (E10) {};
\TransLink{T10}{$Cmd.P_{[3..0]} = 0 (Aucun)$\\$ + $\\$P_{[3..0]} > 5 (Aucun)$}{E10}

\draw[->] (T10) |- (MODIF);

%\EtapeMacroLink{E11}{Modif}{T10}
%\TransLink{T11}{1}{E11}
%\EtapeLink{E12}{Attente\\Commande}{T11}

\draw (I2) -- ($(I2)+(-6,0)$) node[inner sep=0,outer sep=0] (E20) {};
\TransActionLink{T20}{$Cmd.P_{[3..0]} = 1 $ $(K_p)$ \\ $SK_p$ := $R_{Data}$ }{E20}
\draw[->] (T20) |- (MODIF);
%\EtapeMacroLink{E21}{Modif}{T20}
%\TransLink{T21}{1}{E21}
%\EtapeLink{E22}{Attente\\Commande}{T21}


\draw (I2) -- ($(I2)+(-1.7,0)$) node[inner sep=0,outer sep=0] (E30) {};
\TransActionLink{T30}{$Cmd.P_{[3..0]} = 2 $ $(T_d$) \\ $ST_d$ := $R_{Data}$}{E30}
\draw[->] (T30) |- (MODIF);
%\EtapeLink{E31}{}{T30}
%\TransActionLink{T31}{Fin Réception\\$ST_d$ := $R_{Data}$}{E31}
%\EtapeLink{E32}{Attente\\Commande}{T31}
%\TransLink{T32}{/Modif}{E32}
%\EtapeLink{E33}{Attente\\Commande}{T32}

\draw (I2) -- ($(I2)+(+2.6,0)$) node[inner sep=0,outer sep=0] (E40) {};
\TransActionLink{T40}{$Cmd.P_{[3..0]} = 3 $ $(T_e)$ \\$ST_e$ := $R_{Data}$}{E40}
\draw[->] (T40) |- (MODIF);
%\EtapeLink{E41}{}{T40}
%\TransActionLink{T41}{Fin Réception\\$ST_e$ := $R_{Data}$}{E41}
%\EtapeLink{E42}{Attente\\Commande}{T41}
%\TransLink{T42}{/Modif}{E42}
%\EtapeLink{E43}{Attente\\Commande}{T42}

\draw (I2) -- ($(I2)+(+6.9,0)$) node[inner sep=0,outer sep=0] (E50) {};
\TransActionLink{T50}{$Cmd.P_{[3..0]} = 4 $ $(Consigne)$\\$SConsigne$ := $R_{Data}$}{E50}
\draw[->] (T50) |- (MODIF);
%\EtapeLink{E51}{}{T50}
%\TransActionLink{T51}{Fin Réception\\$SConsigne$ := $R_{Data}$}{E51}
%\EtapeLink{E52}{Attente\\Commande}{T51}
%\TransLink{T52}{/Modif}{E52}
%\EtapeLink{E53}{Attente\\Commande}{T52}

\draw (I2) -- ($(I2)+(+12,0)$) node[inner sep=0,outer sep=0] (E60) {};
\TransActionLink{T60}{$Cmd.P_{[3..0]} = 5 $ $(Nb_{Acq})$\\$SNb_{Acq}$ := $R_{Data}$}{E60}
\draw[->] (T60) |- (MODIF);
%\EtapeLink{E61}{}{T60}
%\TransActionLink{T61}{Fin Réception\\$SNb_{Acq}$ := $R_{Data}$}{E61}
%\EtapeLink{E62}{Attente\\Commande}{T61}
%\TransLink{T62}{/Modif}{E62}
%\EtapeLink{E63}{Attente\\Commande}{T62}
\end{tikzpicture}
\end{center}
\end{landscape}

\part{Conception fonctionnelle}
\section{Décomposition fonctionnelle}

\begin{center}
\begin{turn}{90}
\tikzset{
VPH/.style={inner sep=0,minimum width=0.4cm,minimum height=0.2cm,
path picture={\draw[-,black] (path picture bounding box.west) -- (path picture bounding box.east);
\draw[-,black] (path picture bounding box.north west) -- (path picture bounding box.south west);
\draw[-,black] (path picture bounding box.north east) -- (path picture bounding box.south east);}},
VPV/.style={inner sep=0,minimum height=0.4cm,minimum width=0.2cm,
path picture={\draw[-,black] (path picture bounding box.north) -- (path picture bounding box.south);
\draw[-,black] (path picture bounding box.north west) -- (path picture bounding box.north east);
\draw[-,black] (path picture bounding box.south west) -- (path picture bounding box.south east);}}}

\shorthandoff{:}
\begin{tikzpicture}[font=\small]


\node[draw,align=center,inner sep=10pt] (CORR) {Correction\\Asservissement};

\foreach \id in {-2,...,1}
{
\coordinate (CORR-E\id) at ($(CORR.east)+(0,\id*0.3cm)$);
}


\node at ($(CORR)+(5,-2.4)$) [draw,align=center,inner sep=10pt] (COMM) {Gestion\\Communication};
\foreach \id in {-1,1}
{
\coordinate (COMM-S\id) at ($(COMM.south)+(\id*0.3cm,0)$);
}
\foreach \id in {-2,...,2}
{
\coordinate (COMM-N\id) at ($(COMM.north)+(\id*0.3cm,0)$);
}
\node at ($(CORR)+(0,-4)$) [draw,align=center,inner sep=10pt] (CLK) {Clock};

\foreach \id in {-1,1}
{
\coordinate (CLK-E\id) at ($(CLK.east)+(0,\id*0.2cm)$);
}


\node[draw,inner sep=20pt,fit=(CORR) (COMM) (CLK)](SYS) {};



\node at ($(CORR.north)+(0,2.5)$) [ellipse,decoration={bumps,mirror},decorate,draw,align=center] 
(CONV)
{Convertisseur \\ Analogique-Numérique};

\node at ($(CORR.west)+(-5,0)$) [ellipse,decoration={bumps,mirror},decorate,align=center,draw] 
(BOB)
{Pilotage Bobine \\ Pulse Width Modulation};

\node at ($(COMM)+(5,0)$) [ellipse,decoration={bumps,mirror},inner sep=0.3cm,decorate,draw]
(USER)
{Utilisateur};

%\draw[dashed] (CORR) -- node[solid,	isosceles triangle,draw,pos=0.5,rotate=180,fill=white,minimum size=3mm,inner sep=0] {} (BOB);



\path (CORR) -- node[VPV,label={[label distance = 0.2cm]above left:Out}] (mid) {} (BOB);
\draw[->] (CORR) -- (mid.center);
\draw[->] (mid.center) -- (BOB);



\path (COMM) |- node[VPV,label={[label distance = 0.1cm,pos=0.6]above left:Out},pos=0.6] (mid) {} (CORR);
\draw[->] (COMM) |- (mid.center);
\draw[->] (mid.center) -- (CORR);

\path (COMM-S-1) |- node[VPH,label={[label distance = 0.1cm,pos=0.3] left:Te},pos=0.3] (mid) {} (CLK-E1);
\draw[->] (COMM-S-1) -- (mid.center);
\draw[->] (mid.center) |- (CLK-E1);


\draw[dashed] (CLK) -- node[solid,isosceles triangle,draw,pos=0.5,rotate=90,fill=white,minimum size=3mm,inner sep=0,label={[label distance = 0.2cm,pos=0.5]below:$Te_{clk}$}] {} (CORR);

\draw[dashed] (CLK-E-1) -| node[solid,isosceles triangle,draw,pos=0.3,fill=white,minimum size=3mm,inner sep=0,label={[label distance = 0.2cm,pos=0.3]below:$Te_{clk}$}] {} (COMM-S1);

\end{tikzpicture}
\end{turn}
\end{center}


\end{document}
