% !TEX TS-program = pdflatex
% !TEX encoding = UTF-8 Unicode

%%% DOCUMENT DEFINITION
\documentclass[11pt, french]{article} % use larger type; default would be 10pt
\usepackage[utf8]{inputenc} % set input encoding (not needed with XeLaTeX)

%%% PAGE DIMENSIONS
\usepackage{geometry} % to change the page dimensions
\geometry{a4paper} % or letterpaper (US) or a5paper or....
\geometry{margin=1in} % for example, change the margins to 2 inches all round

%%% PACKAGES
\usepackage{graphicx} % support the \includegraphics command and options
\usepackage{booktabs} % for much better looking tables
\usepackage{array} % for better arrays (eg matrices) in maths
\usepackage{paralist} % very flexible & customisable lists (eg. enumerate/itemize, etc.)
\usepackage{verbatim} % adds environment for commenting out blocks of text & for better verbatim
\usepackage{subfig} % make it possible to include more than one captioned figure/table in a single float
\usepackage[frenchb]{babel}

% Package pour schémas MCSE
\usepackage{tikz}
\usetikzlibrary{calc,backgrounds}
\usetikzlibrary{arrows,positioning} 
\usetikzlibrary{decorations,decorations.pathmorphing} 
\usetikzlibrary{shapes,snakes}


% Schémas individuels MCSE
\graphicspath{{../MCSE/}}

%%% HEADERS & FOOTERS
%\usepackage{fancyhdr} % This should be set AFTER setting up the page geometry
%\pagestyle{fancy} % options: empty , plain , fancy

% Rapport projet pluridisciplinaire
% : Xavier Galzin, Stanislas Bertrand, Romain Desille, Frédéric Meslin

\title{Projet pluridisciplinaire : sustentation magnétique}
\author{ Xavier GALZIN, Stan BERTRAND, Romain DESILLE, Fred MESLIN}
\date{30/01/2012}

\begin{document}
\maketitle
\pagebreak

\shorthandoff{:}
\newcommand{\Etape}[2]{
\node[Etape,label={right:#2}] (#1) {};
}
\newcommand{\EtapeLink}[3]{
\node[Etape,below of=#3,label={[align=left]right:#2}] (#1) {};
\draw[->] (#3) -- (#1);
}
\newcommand{\TransLink}[3]{
\node[Trans,align=center,below of=#3] (#1) {};
\node[align=center,anchor=west,at =(#1.east),xshift=10pt] {#2};
\draw[->] (#3) -- (#1);
}
\newcommand{\TransActionLink}[3]{
\node[Trans,align=center,below of=#3] (#1) {};
\node[Line,align=center,anchor=west,at =(#1.east),xshift=10pt] {#2};
\draw[->] (#3) -- (#1);
}



	\tikzset{every node/.style={font=\small, node distance =  0.6cm}}
	\tikzset{Etape/.style={draw,black,minimum size=0.6cm,inner sep=0pt,circle,font=\tiny}}
	\tikzset{Trans/.style={draw,black,minimum width=0.6cm,minimum height=0.0cm,inner sep=0pt,outer sep=0pt,rectangle,font=\tiny}}
	
	%\tikzset{Line/.style={path picture={\draw[black] (path picture bounding box.north west) -- (path picture bounding box.north east);}}}
	\tikzset{Line/.style={path picture={\draw[black] (path picture bounding box.west) -- (path picture bounding box.east);}}}
\part{Spécification fonctionnelle}

\section{Description des entités extérieures}

\subsection{Listes des entités}

\paragraph{} Dans cette partie, nous allons expliciter la décomposition en entités que nous avons retenue :

\vspace{0.2in}
\hspace{0.2in}
\underline{Convertisseur Analogique-Numérique}
\begin{description}
\item[Sortie :] Sample, information, entier
\item[Sortie :] S\_Available, évènement
\item[Entrée :] S\_Query, évènement
\end{description}

\vspace{0.2in}
\hspace{0.2in}
\underline{Pilotage Bobine (PWM)}
\begin{description}
\item[Entrée : ] Out, donnée permanente, entier
\end{description}

\vspace{0.2in}
\hspace{0.2in}
\underline{Utilisateur}   
\begin{description}
\item[Sortie : ] Rx, donnée permanente, bit
\item[Entrée : ] Tx, donnée permanente, bit
\end{description}


\subsection{Graphes des entités}
\vspace{0.2in}

\begin{minipage}[t]{8cm}
\underline{Convertisseur Analogique-Numérique}
\vspace{-0.5in}
\shorthandoff{:}
\newcommand{\Etape}[2]{
\node[Etape,label={right:#2}] (#1) {};
}
\newcommand{\EtapeLink}[3]{
\node[Etape,below of=#3,label={right:#2}] (#1) {};
\draw[->] (#3) -- (#1);
}
\newcommand{\TransLink}[3]{
\node[Trans,align=center,below of=#3] (#1) {};
\node[align=center,anchor=west,at =(#1.east),xshift=10pt] {#2};
\draw[->] (#3) -- (#1);
}
\newcommand{\TransActionLink}[3]{
\node[Trans,align=center,below of=#3] (#1) {};
\node[Line,align=center,anchor=west,at =(#1.east),xshift=10pt] {#2};
\draw[->] (#3) -- (#1);
}




	\tikzset{Etape/.style={draw,black,minimum size=0.7cm,inner sep=0pt,circle,font=\tiny}}
	\tikzset{Trans/.style={draw,black,minimum width=0.7cm,minimum height=0.0cm,inner sep=0pt,outer sep=0pt,rectangle,font=\tiny}}
	
	%\tikzset{Line/.style={path picture={\draw[black] (path picture bounding box.north west) -- (path picture bounding box.north east);}}}
	\tikzset{Line/.style={path picture={\draw[black] (path picture bounding box.west) -- (path picture bounding box.east);}}}

\begin{tikzpicture}\shorthandoff{:}

\Etape{I}{}
\TransLink{T1}{S\_query}{I}
\EtapeLink{S}{Sampling}{T1}
\TransActionLink{T2}{Fin Sampling\\Sample:=Data, S\_available}{S}
\draw[->] (T2) edge [bend left=160,distance=2.5cm] (I);


\end{tikzpicture}









\end{minipage}
\hspace{0.2in}
\begin{minipage}[t]{8cm}
\underline{Pilotage Bobine (PWM)}

\begin{tikzpicture}

\node[Etape,label=left:Génétation PWM] (I) {};

\node[Trans,node distance = 2cm,align=center,right of=I] (T) {};
\node[Line,align=center,anchor=west,at =(T.east),xshift=10pt] {Volonté de Rafraichir\\PWM:=Out};

\draw[->] (I) edge [bend left=90] (T);
\draw[->] (T) edge [bend left=90] (I);


\end{tikzpicture}
\end{minipage}

\newpage
\hspace{0.2in}
\underline{Utilisateur} 

\hspace{0.3in}
Schéma de communication

\begin{tikzpicture}
\Etape{I}{$T_x:=Bit_{Stop}$}
\TransLink{T1}{Volonté d'envoyer}{I}
\EtapeLink{S}{$T_x:=Bit_{Start}$}{T1}
\TransActionLink{T2}{$\uparrow$ I$_{clk}$ \\ $i:=8$}{S}
\EtapeLink{D}{\emph{Bit de donnée}}{T2}
%\tikzstyle{every label}={draw,red,font=\tiny}

\node[Trans,node distance = 1.5cm,left of=D] (T) {};
\node[Line,align=center,anchor=east,at =(T.west),xshift=-10pt] (V) {$\uparrow$ I$_{clk}$ $\cdot$ $i>0$\\[1em]$T_x:=T_{Data}[i]$, $i$-\,- };
\draw[->] (D) edge [out=225,in=-90] (T);
\draw[->] (T) edge [out=90,in=135] (D);

\TransLink{T3}{$\uparrow$ I$_{clk}$ $\cdot$ $i=0$}{D}
\EtapeLink{S}{$T_x:=Bit_{Stop}$}{T3}
\TransActionLink{T4}{$\uparrow$ I$_{clk}$ \\ /Fin Transmition}{S}

\draw[-] (T4) edge [out=-90,in=-90] ($(V.west)-(10pt,0)$);
\draw[->] ($(V.west)-(10pt,0)$) edge [out=90] (I);
\end{tikzpicture}

% reception des données

\begin{tikzpicture}%[every node/.style={font=\tiny}]
\Etape{I}{Attente}
\TransActionLink{T1}{$\uparrow$ I$_{clk}$ $\cdot$ $R_x=Bit_{Start}$ \\ $i:=8$ }{I}
\EtapeLink{D}{\emph{Bit de donnée}}{T1}
%\tikzstyle{every label}={draw,red,font=\tiny}

\node[Trans,node distance = 1.5cm,left of=D] (T) {};
\node[Line,align=center,anchor=east,at =(T.west),xshift=-10pt] (V) {$\uparrow$ I$_{clk}$ $\cdot$ $i>0$\\[1em]$R_{Data}[i]:=R_x$, $i$-\,- };
\draw[->] (D) edge [out=225,in=-90] (T);
\draw[->] (T) edge [out=90,in=135] (D);

\TransLink{T3}{$\uparrow$ I$_{clk}$ $\cdot$ $i=0$}{D}
\EtapeLink{S}{Attente}{T3}
\TransActionLink{T4}{$R_x=Bit_{Stop}$\\ /Fin Réception}{S}

\draw[-] (T4) edge [out=-90,in=-90] ($(V.west)-(10pt,0)$);
\draw[->] ($(V.west)-(10pt,0)$) edge [out=90] (I);
\end{tikzpicture}

% Schema de communication

\begin{tikzpicture}%[every node/.style={font=\tiny}]

\node (T) {};

\TransLink{T1}{Volonté Lecture}{T}

\EtapeLink{E1}{$T_{Data}:=Commande$, /Volonté d'envoyer}{T1}

\TransLink{T2}{Fin Transmition}{E1}

\EtapeLink{E2}{Lecture}{T2}

\TransActionLink{T3}{Fin Réception \\ $R_{Data}$}{E2}

\EtapeLink{E3}{}{T3}

	\draw[-] (E3) edge [out=160, in=90] (T.south);


\TransLink{T1}{Volonté Ecriture}{E3}

\EtapeLink{E1}{$T_{Data}:=Commande$, /Volonté d'envoyer}{T1}

\TransLink{T2}{Fin Transmition}{E1}

\EtapeLink{E2}{$T_{Data}:=Donnee$, /Volonté d'envoyer}{T2}

\TransLink{T3}{Fin Transmition}{E2}

	\draw[->] (T3) edge [out=-90, in=210,out distance=4cm] (E3);

\end{tikzpicture}


Les fonctions de Transmition/Reception, sont respectivement representé avec T\_x et R\_x, or les deux péripheriques sont reliés avec un cable croisé. Le T\_x est donc en realité le R\_x du systeme et inversement.

\hspace{0.3in}
Fonction Reception

%% reception des données

\begin{tikzpicture}[every node/.style={font=\footnotesize, node distance =  0.7cm}]
\Etape{I}{Attente}
\TransActionLink{T1}{$\uparrow$ I$_{clk}$ $\cdot$ $R_x=Bit_{Start}$ \\ $i:=8$ }{I}
\EtapeLink{D}{\emph{Bit de donnée}}{T1}
%\tikzstyle{every label}={draw,red,font=\tiny}

\node[Trans,node distance = 1cm,left of=D] (T) {};
\node[Line,align=center,anchor=east,at =(T.west),xshift=-10pt] (V) {$\uparrow$ I$_{clk}$ $\cdot$ $i>0$\\[1em]$R_{Data}[i]:=R_x$, $i$-\,- };
\draw[->] (D) edge [out=225,in=-90] (T);
\draw[->] (T) edge [out=90,in=135] (D);

\TransLink{T3}{$\uparrow$ I$_{clk}$ $\cdot$ $i=0$}{D}
\EtapeLink{S}{Attente}{T3}
\TransActionLink{T4}{$R_x=Bit_{Stop}$\\ /Fin Réception}{S}

\draw[-] (T4) edge [out=-90,in=-90] ($(V.west)-(10pt,0)$);
\draw[->] ($(V.west)-(10pt,0)$) edge [out=90] (I);
\end{tikzpicture}

\hspace{0.3in}
Fonction Transmition


\begin{tikzpicture}
\Etape{I}{$T_x:=Bit_{Stop}$}
\TransLink{T1}{Volonté d'envoyer}{I}
\EtapeLink{S}{$T_x:=Bit_{Start}$}{T1}
\TransActionLink{T2}{$\uparrow$ I$_{clk}$ \\ $i:=8$}{S}
\EtapeLink{D}{\emph{Bit de donnée}}{T2}
%\tikzstyle{every label}={draw,red,font=\tiny}

\node[Trans,node distance = 1cm,left of=D] (T) {};
\node[Line,align=center,anchor=east,at =(T.west),xshift=-10pt] (V) {$\uparrow$ I$_{clk}$ $\cdot$ $i>0$\\[1em]$T_x:=T_{Data}[i]$, $i$-\,- };
\draw[->] (D) edge [out=225,in=-90] (T);
\draw[->] (T) edge [out=90,in=135] (D);

\TransLink{T3}{$\uparrow$ I$_{clk}$ $\cdot$ $i=0$}{D}
\EtapeLink{S}{$T_x:=Bit_{Stop}$}{T3}
\TransActionLink{T4}{$\uparrow$ I$_{clk}$ \\ /Fin Transmition}{S}

\draw[-] (T4) edge [out=-90,in=-90] ($(V.west)-(10pt,0)$);
\draw[->] ($(V.west)-(10pt,0)$) edge [out=90] (I);
\end{tikzpicture}

\part{Conception fonctionnelle}
\section{Délimitation des E/S du système}
\section{Décomposition fonctionnelle}

%\begin{tikzpicture}\shorthandoff{:}

\node[inner sep=1cm,rounded corners,fill=gray!50]
(SYS)
{Système};

\node[above of=SYS,node distance=4cm,ellipse,decoration={bumps,mirror},decorate,draw,align=center] 
(CONV)
{Convertisseur \\ Analogique-Numérique};

\node[left of=SYS,node distance=6cm,ellipse,decoration={bumps,mirror},decorate,align=center,draw] 
(BOB)
{Pilotage Bobine \\ Pulse Width Modulation};

\node[right of=SYS,node distance=6cm,ellipse,decoration={bumps,mirror},inner sep=0.3cm,decorate,draw]
(USER)
{Utilisateur};

\draw [->>,very thick] (SYS) -- node[above,pos=0.5]{Out} (BOB) ;

\draw [<-, thick] (CONV) edge[dashed,bend right=35] node[left,pos=0.5]{S\_Qerry} (SYS) ;
\draw [->>,very thick] (CONV) edge[bend right=15] node[right,pos=0.5]{Sample} (SYS) ;
\draw [->, thick] (CONV) edge[dashed,bend left=35] node[right,pos=0.5]{S\_available} (SYS) ;




\draw [->>,very thick] (SYS) edge[bend left=10] node[above,pos=0.5]{Tx} (USER) ;
\draw [<<-,very thick] (SYS) edge[bend left=-10] node[below,pos=0.5]{Rx} (USER) ;


\end{tikzpicture}







\end{document}
