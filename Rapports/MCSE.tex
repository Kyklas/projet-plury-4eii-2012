% !TEX TS-program = pdflatex
% !TEX encoding = UTF-8 Unicode

%%% DOCUMENT DEFINITION
\documentclass[11pt, french]{article} % use larger type; default would be 10pt
\usepackage[utf8]{inputenc} % set input encoding (not needed with XeLaTeX)

%%% PAGE DIMENSIONS
\usepackage{geometry} % to change the page dimensions
\geometry{a4paper} % or letterpaper (US) or a5paper or....
\geometry{margin=1in} % for example, change the margins to 2 inches all round

%%% PACKAGES
\usepackage{graphicx} % support the \includegraphics command and options
\usepackage{booktabs} % for much better looking tables
\usepackage{array} % for better arrays (eg matrices) in maths
\usepackage{paralist} % very flexible & customisable lists (eg. enumerate/itemize, etc.)
\usepackage{verbatim} % adds environment for commenting out blocks of text & for better verbatim
\usepackage{subfig} % make it possible to include more than one captioned figure/table in a single float
\usepackage[frenchb]{babel}

% Package pour schémas MCSE
\usepackage{tikz}
\usetikzlibrary{calc,backgrounds}


% Schémas individuels MCSE
\graphicspath{{../MCSE/}}

%%% HEADERS & FOOTERS
%\usepackage{fancyhdr} % This should be set AFTER setting up the page geometry
%\pagestyle{fancy} % options: empty , plain , fancy

% Rapport projet pluridisciplinaire
% : Xavier Galzin, Stanislas Bertrand, Romain Desille, Frédéric Meslin

\title{Projet pluridisciplinaire : sustentation magnétique}
\author{ Xavier GALZIN, Stan BERTRAND, Romain DESILLE, Fred MESLIN}
\date{30/01/2012}

\begin{document}
\maketitle
\pagebreak


\part{Spécification fonctionnelle}

\section{Description des entités extérieures}
\paragraph{} Dans cette partie, nous allons expliciter la décomposition en entités que nous avons retenue :

Convertisseur Analogique-Numérique
\begin{description}
\item[Sortie : ]Sample, donnée permanente sur 10 bits
\item[Sortie : ]S\_available, évènement
\item[Entrée : ]S\_Query, évènement
\end{description}

Utilisateur
\begin{description}
\item[Sortie : ]Rx, donnée permanente, bit
\item[Entrée : ]Tx, donnée permanente, bit
\end{description}

Pilotage Bobine (PWM)
\begin{description}
\item[Entrée : ]Out, donnée permanente sur 8 bits
\end{description}



\subsection{Listes des entités}
\subsection{Graphes des entités}
\section{Délimitations des E/S du système}
\section{Spécification fonctionnelle}



\part{Conception fonctionnelle}
\section{Délimitation des E/S du système}
\section{Décomposition fonctionnelle}




\shorthandoff{:}
\newcommand{\Etape}[2]{
\node[Etape,label={right:#2}] (#1) {};
}
\newcommand{\EtapeLink}[3]{
\node[Etape,below of=#3,label={right:#2}] (#1) {};
\draw[->] (#3) -- (#1);
}
\newcommand{\TransLink}[3]{
\node[Trans,align=center,below of=#3] (#1) {};
\node[align=center,anchor=west,at =(#1.east),xshift=10pt] {#2};
\draw[->] (#3) -- (#1);
}
\newcommand{\TransActionLink}[3]{
\node[Trans,align=center,below of=#3] (#1) {};
\node[Line,align=center,anchor=west,at =(#1.east),xshift=10pt] {#2};
\draw[->] (#3) -- (#1);
}




	\tikzset{Etape/.style={draw,black,minimum size=0.7cm,inner sep=0pt,circle,font=\tiny}}
	\tikzset{Trans/.style={draw,black,minimum width=0.7cm,minimum height=0.0cm,inner sep=0pt,outer sep=0pt,rectangle,font=\tiny}}
	
	%\tikzset{Line/.style={path picture={\draw[black] (path picture bounding box.north west) -- (path picture bounding box.north east);}}}
	\tikzset{Line/.style={path picture={\draw[black] (path picture bounding box.west) -- (path picture bounding box.east);}}}

\begin{tikzpicture}\shorthandoff{:}

\Etape{I}{}
\TransLink{T1}{S\_query}{I}
\EtapeLink{S}{Sampling}{T1}
\TransActionLink{T2}{Fin Sampling\\Sample:=Data, S\_available}{S}
\draw[->] (T2) edge [bend left=160,distance=2.5cm] (I);


\end{tikzpicture}

\begin{tikzpicture}

\node[Etape,label=left:Génétation PWM] (I) {};

\node[Trans,node distance = 2cm,align=center,right of=I] (T) {};
\node[Line,align=center,anchor=west,at =(T.east),xshift=10pt] {Volonté de Rafraichir\\PWM:=Out};

\draw[->] (I) edge [bend left=90] (T);
\draw[->] (T) edge [bend left=90] (I);


\end{tikzpicture}

\begin{tikzpicture}


\node[ellipse,decoration={bumps,mirror},decorate,minimum height=2cm,minimum width=6cm,draw] {Ici};

\end{tikzpicture}





%\vspace{-1.5cm}
\hspace{-4cm}
\begin{tikzpicture}%[every node/.style={font=\tiny}]



\Etape{T}{}

\draw (T) -- ($(T)+(-4,0)$) node[inner sep=0,outer sep=0] (E3) {};

\TransLink{T1}{Volonté Ecriture}{E3}

\EtapeLink{E1}{$T_{Data}:=Commande$, /Volonté d'envoyer}{T1}

\TransLink{T2}{Fin Transmition}{E1}

\EtapeLink{E2}{$T_{Data}:=Donnee$, /Volonté d'envoyer}{T2}

\TransLink{T3}{Fin Transmition}{E2}

\draw[->] (T3) edge [out=-90, in=160,out distance=4cm,in distance=9cm] node[inner sep=0,outer sep=0,minimum width=0cm,pos=0.2] (RET){} (T);


\draw (T) -- ($(T)+(4,0)$) node[inner sep=0,outer sep=0] (E1) {};

\TransLink{T1}{Volonté Lecture}{E1}

\EtapeLink{E1}{$T_{Data}:=Commande$, /Volonté d'envoyer}{T1}

\TransLink{T2}{Fin Transmition}{E1}

\EtapeLink{E2}{Lecture}{T2}

\TransActionLink{T3}{Fin Réception \\ $R_{Data}$}{E2}


\draw[->] (T3) edge [out=-90,in=-40,out distance=1cm] (RET.center);


\end{tikzpicture}
\vspace{-2cm}
%\vspace{-0.4cm}
\begin{tikzpicture}

\node[Etape,label=right:Génétation PWM] (I) {};
\TransActionLink{T}{Volonté de Rafraichir\\PWM:=Out}{I}

%\node[Trans,node distance = 2cm,align=center,right of=I] (T) {};
%\node[Line,align=center,anchor=west,at =(T.east),xshift=10pt] {Volonté de Rafraichir\\PWM:=Out};


\draw[->] (T) edge [bend left=160,distance=2.5cm] (I);

%\draw[->] (I) edge [bend left=90] (T);
%\draw[->] (T) edge [bend left=90] (I);

\end{tikzpicture}

\end{document}
