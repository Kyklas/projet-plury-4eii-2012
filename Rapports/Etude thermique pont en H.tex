% !TEX TS-program = pdflatex
% !TEX encoding = UTF-8 Unicode

%%% DOCUMENT DEFINITION
\documentclass[11pt, french]{article} % use larger type; default would be 10pt
\usepackage[utf8]{inputenc} % set input encoding (not needed with XeLaTeX)

%%% PAGE DIMENSIONS
\usepackage{geometry} % to change the page dimensions
\geometry{a4paper} % or letterpaper (US) or a5paper or....
\geometry{margin=1in} % for example, change the margins to 2 inches all round

%%% PACKAGES
\usepackage{graphicx} % support the \includegraphics command and options
\usepackage{booktabs} % for much better looking tables
\usepackage{array} % for better arrays (eg matrices) in maths
\usepackage{paralist} % very flexible & customisable lists (eg. enumerate/itemize, etc.)
\usepackage{verbatim} % adds environment for commenting out blocks of text & for better verbatim
\usepackage{subfig} % make it possible to include more than one captioned figure/table in a single float
\usepackage[frenchb]{babel}


%%% HEADERS & FOOTERS
%\usepackage{fancyhdr} % This should be set AFTER setting up the page geometry
%\pagestyle{fancy} % options: empty , plain , fancy

% Rapport projet pluridisciplinaire : etude thermique du pont en H
% : Xavier Galzin, Stanislas Bertrand, Romain Desille, Frédéric Meslin

\title{Projet pluridisciplinaire : Etude Thermique du pont en H}
\author{Xavier GALZIN, Stanislas BERTRAND, Romain DESILLE, Frédéric MESLIN}
\date{7/03/2012}

\begin{document}
\maketitle

\pagebreak

\section{Données}

\noindent
D'après la Datasheet du A4950 :

\vspace{0.5cm}

\noindent
$ 
R_{TH} = 62 \char123C/W  \\
T_{JUNC\_MAX} = 160 \char123C  \\
T_{AMB} = 25\char123C  \\
R_{DSON} = 1.3 \Omega  \\
$



\noindent
A noter que cette valeur de $R_{TH}$ est donnée dans le cas l'on a aurait des plages de cuivres d'environ 2cm par 2cm de chaque côté du composant. De plus, la valeur de $R_{DSON}$ est une valeur pire cas.

\section{Equations}

\noindent
L'équation de la température de la jonction est donnée par :

\vspace{0.5cm}

\noindent
$
T_{JUNC} = R_{TH} * P_{DIS} + T_{AMB}
$

\vspace{0.5cm}

\noindent
P étant la puissance \textbf{dissipée} dans le composant que l'on calculera par la formule :

\vspace{0.5cm}

\noindent
$
P_{DIS} = R_{DSON} * I\up{2} = 1.3 * 1.28\up{2} = 5.2 W
$

\vspace{0.5cm}

\noindent
Finalement, on obtient :

\vspace{0.5cm}

\noindent
$
T_{JUNC} = 62 * 5.2 + 25 = 157.06 \char123C
$

\section{Conclusion}

\noindent
Le résultat trouvé est très proche de la valeur limite du composant. Cependant, nous avons pris des paramètres extrèmes qui ne seront normalent pas atteint. De plus, nous comptons augmenter la surface de cuivre qui assurera la dissipation afin d'obtenir une marge plus grande par rapport à ce résultat.


\end{document}
